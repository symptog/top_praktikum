\subsection{Java Universal Network/Graph Framework}
\begin{frame}
	\frametitle{Java Universal Network/Graph Framework}
	\begin{itemize}
		\item Java Framework zur Modellierung und Analyse von Graphen
		\item Zur Einbettung in Java Swing Anwendungen gedacht
		\item Implementiert einige Algorithmen, aber kein Growing Neural Gas
		\item Definition eigener Objekte als Knoten möglich
		\item Kann GraphML (XML für Graphen) einlesen
		\item Implementiert bereits Maus-Gesten zur Steuerung
		\item Weitere Interaktivität muss in der Anwendung selbst geschaffen werden
	\end{itemize}
\end{frame}

\subsection{JGraphT}
\begin{frame}
	\frametitle{JGraphT}
	\begin{itemize}
		\item Kein Visualisierungsframework
		\item Framework zur Arbeit mit Graphenalgorithmen
		\item Bietet verschiedenen Algorithmen aber kein Growing Neural Gas
		\item Definition eigener Objekte als Knoten möglich
		\item Syntax ähnlich dem JUNG Framework
		\item Visualisierung möglich über JGraphX
	\end{itemize}
\end{frame}

\subsection{JGraphX}
\begin{frame}
	\frametitle{JGraphX}
	\begin{itemize}
		\item Kommerzielles Framework zur Visualisierung von Graphen 
		\item Implementiert keine Algorithmen
		\item Definition eigener Objekte als Knotenwert möglich
		\item Zur Einbettung in Java Swing Anwendungen gedacht
		\item Interaktivität muss in der Anwendung selbst geschaffen werden
	\end{itemize}
\end{frame}

\subsection{Webbasierte Werkzeuge}
\subsubsection{D3.js}
\begin{frame}
	\frametitle{D3.js}
	\begin{itemize}
		\item Framework zur Erstellung von Diagrammen und Graphen
		\item Bietet viele Layouts und Animation
		\item Overpowered für Graphendarstellung
		\item Demograph: \url{http://bl.ocks.org/mbostock/1093130}
		\item Kartogramm: \url{http://bl.ocks.org/mbostock/4060606}
	\end{itemize}
\end{frame}
\subsubsection{Sigma.js}
\begin{frame}
	\frametitle{Sigma.js}
	\begin{columns}
	\begin{column}{0.49\textwidth}
	\begin{itemize}
		\item Reines Werkzeug zur Erstellung von Graphen
		\item Einfache Syntax
		\item Erweiterbar durch Plugins
		\item Kann GraphML (XML für Graphen) einlesen
	\end{itemize}
	\end{column}
	\begin{column}{0.49\textwidth}
	\begin{figure}[h!]
		\centering
		\includegraphics[width=\textwidth]{bilder/sigma2.png}
		\caption{Sigma.js}
		\label{fig:sigma}
	\end{figure}
	\end{column}
	\end{columns}

\end{frame}
