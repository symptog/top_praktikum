\subsection{DemoGNG}
\begin{frame}
	\frametitle{DemoGNG}
	\begin{itemize}
		\item Von Bernd Fritzke entwickelt
		\item Visualisierung und Steuerung mehrerer verschiedener Algorithmen
		\item Implementierung als Java-Applet unter Zuhilfenahme der 
		Grafikbibliotheken Abstract Window Toolkit und Swing
		\item Vordefinierte Eingabedaten
		\item Einstellung des Algorithmus über Eingabefelder möglich
		\item Demo: \url{http://www.demogng.de/}
	\end{itemize}
	
\end{frame}

\subsection{Matlab Implementation}
\begin{frame}
	\frametitle{Matlab Implementation}
	\begin{itemize}
		\item Keine eigenständige Anwendung
		\item Setzt zwingend Matlab voraus
		\item Eingabedaten müssen separat erstellt werden
		\item Einstellung des Algorithmus im Quelltext
		\item Keine Interaktivität
		\item Algorithmus stürzt regelmäßig ab
	\end{itemize}
\end{frame}
\begin{frame}
	\frametitle{Matlab Implementation}
	\begin{figure}[h!]
		\centering
		\includegraphics[width=0.9\textwidth]{bilder/matlab.png}
		\caption{Visualisierung mit Matlab}
		\label{fig:matlab}
	\end{figure}
\end{frame}

\subsection{Modular toolkit for Data Processing}
\begin{frame}
	\frametitle{Modular toolkit for Data Processing}
	\begin{itemize}
		\item Python Bibliothek
		\item Beinhaltet ca. 150 Algorithmen
		\item Nachteil: \textbf{visualisiert keine Daten}
		\item Einstellung des Algorithmus durch Parameter der Klasse
		\item Einfaches, schrittweises Iterieren über den Algorithmus
		\item Zusätzliche Visualisierung mit Matplotlib
		\item Keine Interaktivität
	\end{itemize}
\end{frame}
\begin{frame}
	\frametitle{Modular toolkit for Data Processing}
	\begin{figure}[h!]
		\centering
		\includegraphics[width=0.5\textwidth]{bilder/mdptoolkit.png}
		\caption{Visualisierung des GNG des MDP-Toolkits}
		\label{fig:mdptoolkit}
	\end{figure}
\end{frame}

\subsection{GoGNG}
\begin{frame}
	\frametitle{GoGNG}
	\begin{itemize}
		\item Algorithmus in Go geschrieben
		\item Einstellung des Algorithmus im Quelltext
		\item Eingabedaten müssen separat erzeugt werden
		\begin{itemize}
			\item CSV Datei mit x,y-Koordinaten
		\end{itemize}
		\item Ausgabedaten im JSON Format
		\item Visualisierung in Python mit Matplotlib und NetworkX
		\item Keine Interaktivität
	\end{itemize}
\end{frame}
\begin{frame}
	\frametitle{GoGNG}
	\begin{figure}[h!]
		\centering
		\subfigure[Eingabedaten]{\includegraphics[width=0.46\textwidth]{bilder/go_1.png}}
		\subfigure[Ausgabegraph]{\includegraphics[width=0.5\textwidth]{bilder/go_2.png}}
		\caption{Growing Neural Gas mit Go}
		\label{fig:go}
	\end{figure}
\end{frame}
